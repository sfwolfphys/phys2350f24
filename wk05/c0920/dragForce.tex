\documentclass[12pt,hidelinks]{article}

\usepackage[margin=1in]{geometry}
\usepackage{amsmath}
\usepackage{graphicx}
\usepackage{cclicenses}
\usepackage{fancyhdr,lastpage}
\usepackage[]{hyperref}
\pagestyle{fancy}
\usepackage{enumitem}
\setlist{nosep}
\setlist[enumerate]{label=(\alph*)}
\usepackage{tcolorbox}
\usepackage{fancyvrb}
\VerbatimFootnotes
\usepackage{tikz}
\usepackage[per-mode=fraction]{siunitx}

\usepackage{minted}

\renewcommand{\thesection}{\Roman{section}.}
\renewcommand{\thesubsection}{\thesection\Alph{subsection}}

\lhead{TUTORIAL: NEWTON'S LAWS: DEALING WITH AIR RESISTANCE IN 2D}
\rhead{}
\lfoot{Tutorial 4, Week 5 \\\cc 2024 East Carolina University}
\cfoot{}
\rfoot{{\bf Page \thepage \hspace*{0.4em}of \pageref{LastPage}}\\Contact: \url{wolfs15@ecu.edu}}

\newcommand{\checkin}{{\bf \noindent $\Rightarrow$ PAUSE and check with an instructor or another
  group.}} 

\begin{document}

\section{Characterizing Air resistance}

Previously, we noted that air resistance had several forms. It could be\ldots
\begin{enumerate}[label=(\roman*)]
  \item proportional to the speed of the object (with magnitude $c_1\left|v\right|$),
  \item proportional to the square of the speed (with magnitude $c_2v^2$), or
  \item a combination of the above two terms (linear and quadratic).
\end{enumerate}
where $c_1$ and $c_2$ are positive constants. Both of these velocity terms are in the
\textit{opposite} direction as the velocity.  Practically speaking, we operate under a few
rules of thumb to determine which terms are most important.
\begin{itemize}
  \item Small, slow objects tend to have $c_1 \neq 0$ and $c_2 \approx 0$
  \item Large, fast objects tend to have $c_1 \approx 0$ and $c_2 \neq 0$
\end{itemize}

Which terms do you think will apply to the motion of a cannonball fired from a cannon?
\vspace{0.5in}

\section{Calculating Air resistance in 2D}

Suppose you have an object moving with a velocity
$\vec{v} = (\SI{30}{\meter\per\second}, \SI{40}{\meter\per\second})$ and the drag coefficients
$c_1 = \SI{0.0001}{\kilo\gram\per\second}$ and $c_2 = \SI{0.01}{\kilo\gram\per\meter}$. These
coefficients are consistent with large objects moving through air.

\begin{enumerate}
  \item Calculate the magnitude of the air resistance force on the object noted above. Consider
  the linear and quadratic terms separately. Is it appropriate to ignore either term?
  \vspace{1in}
  \item Calculate the vector $\hat{v}$ for the velocity vector $\vec{v}$ given above. Does this
  vector have units? What is the length of this vector? What direction is this vector pointing
  in compared to the velocity? What direction is this vector pointing in compared to the air
  resistance?
  \[
    \hat{v} = \frac{\vec{v}}{\left|\vec{v}\right|}
  \]
  \vspace{0.5in}
  \item Calculate the air resistance force in component form $(F_x,F_y)$. You can ignore any
  contributions to this force that are small. \vspace{1.5in}
  \item If you know the velocity components $(v_x,v_y)$, write an algebraic expression allowing
  you to find the air resistance force assuming $c_1=0$. Double check your expression by
  ensuring that you get the same numerical answer as you did for part (c). [Question: Why did I
  make you calculate $\hat{v}$ in part (b)?] \vspace{1in}
\end{enumerate}

\checkin

\section{Projectile motion with drag}

Let's go back to the Kenyan Coast and use the data from Wednesday. To this data, we'll include
the following:
\begin{itemize}
  \item The mass of a cannonball is \SI{1.0}{\kilo\gram}.
  \item The drag coefficients are $c_1 = \SI{3.3e-5}{\kilo\gram\per\second}$ and $c_2
  = \SI{2.9e-3}{\kilo\gram\per\meter}$
\end{itemize}

I have created an app which allows you to generate the motion for the cannonball with the
ability to tune parameters to fit the motion that we are studying.  It is linked to the Canvas
page for today's assignment.

In my code, I defined the following function to calculate the turbulent (quadratic) drag
force. In the space to the right, describe what is being done on each line (beginning with line
7). Convince yourself that this is correctly calculating the drag force.
\begin{minted}[linenos, numbersep=5pt]{python}
def turbDrag(c,v):
    ''' Calculates the turbulent drag force.
    Inputs:
        c = Drag coefficient (positive number)
        v = Velocity vector (two components in x,y directions)
    '''
    [vx, vy] = v
    spd = np.sqrt(vx**2+vy**2)
    vhat = v/spd
    FdragMag = c*spd**2
    Fdrag = - FdragMag * vhat
    return Fdrag
\end{minted}

\subsection{Final task}
\begin{enumerate}
  \item Find the launch angle $\theta$ which gives the largest range for the cannon.
  \item Play with the parameters in the app and figure out something interesting, then write
  out what that is and give the parameters that you used to generate that knowledge. (This
  should be a paragraph or so. It would be great if you got screenshots and put them in a
  document that you uploaded to Canvas with this assignment).
\end{enumerate}

\end{document}
