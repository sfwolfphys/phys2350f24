\documentclass[12pt,hidelinks]{article}

\usepackage[margin=1in]{geometry}
\usepackage{amsmath}
\usepackage{graphicx}
\usepackage{cclicenses}
\usepackage{fancyhdr,lastpage}
\usepackage[]{hyperref}
\pagestyle{fancy}
\usepackage{enumitem}
\setlist{nosep}
\setlist[enumerate]{label=(\alph*)}
\usepackage{tcolorbox}
\usepackage{fancyvrb}
\VerbatimFootnotes
\usepackage{tikz}
\usepackage{siunitx}

\renewcommand{\thesection}{\Roman{section}.}

\lhead{TUTORIAL: NEWTON'S LAWS: DEALING WITH FRICTION}
\rhead{}
\lfoot{Tutorial 1, Week 5 \\\cc 2024 East Carolina University}
\cfoot{}
\rfoot{{\bf Page \thepage \hspace*{0.4em}of \pageref{LastPage}}\\Contact: \url{wolfs15@ecu.edu}}

\newcommand{\checkin}{{\bf \noindent $\Rightarrow$ PAUSE and check with an instructor or another
  group.}} 

\begin{document}

\section{Applying Newton's laws to a system with friction}
Suppose that a block slides down a fixed, inclined plane with $\theta=\SI{30}{\deg}$.  You have
analyzed the material of the block and the plane previously and have found that they have a
coefficient of kinetic friction of $\mu_k = 0.3$.
\begin{enumerate}
  \item In the space below, draw a free-body diagram for the block just after it has been
  released from rest. You should ensure that your diagram is accurate in the direction that the
  forces point, but it need not have accurate force magnitudes.  All important angles should be
  labeled. \vspace{1.5in}
  \item In the space below, sketch a qualitatively correct graph of velocity vs. time $(v$ vs.\
  $t)$ for the block.  You should assume that the block is released from rest at $t=0$ and the
  velocity is positive if it is moving down the ramp.
  \begin{center}
    \begin{tikzpicture}[>=stealth]
      \draw[<->] (-1,0) -- (10,0) node [right]{$t$}; \draw[<->] (0,-1) -- (0,5) node
      [left]{$v$};
    \end{tikzpicture}
  \end{center}
  \item On the same set of axes above, show the $v$ vs. $t$ graph that \textit{would have been}
  correct if there was \textit{no} friction.  Make sure that both graphs are consistent with
  each other and be prepared to defend your rationale to another group/the instructor. \vfill
\end{enumerate}

\checkin
\newpage

\noindent
Now let's see if we can go from qualitative predictions to quantitative predictions.
\begin{enumerate}[resume]
  \item How will you orient your coordinate system for this problem? Show how it will be
  oriented on your free-body diagram on the previous page and defend your choice in the space
  below. [Hint: You should choose your x and y directions so that the \textit{fewest} number of
  forces need to be broken down into coordinates.] \vspace{1in}
  \item Break down your forces into components along each direction of your coordinate system
  of choice.  Relate the net force to the acceleration in each direction and simplify your
  result. \vspace{0.1in}

  \rule{0.5in}{0.5pt}-component \hspace{0.35\textwidth} \rule{0.5in}{0.5pt}-component
  \vspace{2in}
  \item Simplify the equations above and solve for the acceleration of the block $a$.
  \vspace{2in}
  \item Ensure that the previous result matches the $v$ vs.\ $t$ graph that you drew in part
  (b). Resolve any inconsistencies.
\end{enumerate}
\vfill \checkin
\newpage
\section{Applying Newton's laws to a system with friction}
Suppose that a block is at the bottom of a fixed, inclined plane with $\theta=\SI{30}{\deg}$.
You have analyzed the material of the block and the plane previously and have found that they
have a coefficient of kinetic friction of $\mu_k = 0.3$. The block is initially given a quick
shove up the ramp.

\begin{enumerate}
  \item In the space below, draw a free-body diagram for the block just after it has been
  shoved up the ramp. \vspace{1.5in}
  \item In the space below, sketch a qualitatively correct graph of velocity vs. time $(v$ vs.\
  $t)$ for the block.  You should assume that the block is moving at speed $v_0\neq 0$ after the
  shove has finished at $t=0$ and the velocity is positive if it is moving up the ramp.
  \begin{center}
    \begin{tikzpicture}[>=stealth]
      \draw[<->] (-1,0) -- (10,0) node [right]{$t$}; \draw[<->] (0,-1) -- (0,5) node
      [left]{$v$};
    \end{tikzpicture}
  \end{center}
  \item On the same set of axes above, show the $v$ vs. $t$ graph that \textit{would have been}
  correct if there was \textit{no} friction.  Make sure that both graphs are consistent with
  each other and be prepared to defend your rationale to another group/the instructor. \vfill
\end{enumerate}

\checkin
\newpage

\noindent
Now let's see if we can go from qualitative predictions to quantitative predictions.
\begin{enumerate}[resume]
  \item Can you re-use the coordinate system that you had from part \textrm{I}? Show how it
  will be oriented on your free-body diagram on the previous page and defend your choice in the
  space below. \vspace{1in}
  \item Break down your forces into components along each direction of your coordinate system
  of choice.  Relate the net force to the acceleration in each direction and simplify your
  result. \vspace{0.1in}

  \rule{0.5in}{0.5pt}-component \hspace{0.35\textwidth} \rule{0.5in}{0.5pt}-component
  \vspace{2in}
  \item Simplify the equations above and solve for the acceleration of the block $a$.
  \vspace{2in}
  \item Ensure that the previous result matches the $v$ vs.\ $t$ graph that you drew in part
  (b). Resolve any inconsistencies.
\end{enumerate}
\vfill \checkin

\end{document}
